\documentclass{article}
\usepackage{fullpage}
% \usepackage{url}


\usepackage{amssymb,amsmath}
\usepackage{ifxetex,ifluatex}
\ifxetex
  \usepackage{fontspec,xltxtra,xunicode}
  \defaultfontfeatures{Mapping=tex-text,Scale=MatchLowercase}
%  \newcommand{\euro}{€}
\else
  \ifluatex
    \usepackage{fontspec}
    \defaultfontfeatures{Mapping=tex-text,Scale=MatchLowercase}
%    \newcommand{\euro}{€}
  \else
    \usepackage[utf8]{inputenc}
%    \usepackage{eurosym}
  \fi
\fi
\usepackage{color}
\usepackage{fancyvrb}
\DefineShortVerb[commandchars=\\\{\}]{\|}
\DefineVerbatimEnvironment{Highlighting}{Verbatim}{commandchars=\\\{\}}
% Add ',fontsize=\small' for more characters per line
\newenvironment{Shaded}{}{}
\newcommand{\KeywordTok}[1]{\textcolor[rgb]{0.00,0.22,0.06}{\textbf{{#1}}}}
\newcommand{\DataTypeTok}[1]{\textcolor[rgb]{0.28,0.06,0.00}{{#1}}}
\newcommand{\DecValTok}[1]{\textcolor[rgb]{0.12,0.30,0.22}{{#1}}}
\newcommand{\BaseNTok}[1]{\textcolor[rgb]{0.12,0.30,0.22}{{#1}}}
\newcommand{\FloatTok}[1]{\textcolor[rgb]{0.12,0.30,0.22}{{#1}}}
\newcommand{\CharTok}[1]{\textcolor[rgb]{0.12,0.22,0.30}{{#1}}}
\newcommand{\StringTok}[1]{\textcolor[rgb]{0.12,0.22,0.30}{{#1}}}
\newcommand{\CommentTok}[1]{\textcolor[rgb]{0.19,0.30,0.30}{\textit{{#1}}}}
\newcommand{\OtherTok}[1]{\textcolor[rgb]{0.00,0.22,0.06}{{#1}}}
\newcommand{\AlertTok}[1]{\textcolor[rgb]{0.50,0.00,0.00}{\textbf{{#1}}}}
\newcommand{\FunctionTok}[1]{\textcolor[rgb]{0.01,0.08,0.25}{{#1}}}
\newcommand{\RegionMarkerTok}[1]{{#1}}
\newcommand{\ErrorTok}[1]{\textcolor[rgb]{0.50,0.00,0.00}{\textbf{{#1}}}}
\newcommand{\NormalTok}[1]{{#1}}
\ifxetex
  \usepackage[setpagesize=false, % page size defined by xetex
              unicode=false, % unicode breaks when used with xetex
              xetex,
              colorlinks=true,
              linkcolor=blue]{hyperref}
\else
  \usepackage[unicode=true,
              colorlinks=true,
              linkcolor=blue]{hyperref}
\fi
\hypersetup{breaklinks=true, pdfborder={0 0 0}}
\setlength{\parindent}{0pt}
\setlength{\parskip}{6pt plus 2pt minus 1pt}
\setlength{\emergencystretch}{3em}  % prevent overfull lines
\setcounter{secnumdepth}{0}
 
% \EndDefineVerbatimEnvironment{Highlighting}




\begin{document}
\title{Typeful PPX and Value Implicits}
\date{}
\author{Jun Furuse}
\maketitle

\begin{abstract}\label{abstract}

This talk presents Typeful PPX, a novel PPX preprocessing technique with
types. Type dependent preprocessing is fairly easy once PPX is combined
with the compiler type checker.

\texttt{ppx\_implicits} is presented as a demonstration of Typeful PPX,
which provides type dependent implicit values, combining it with the
optional parameters and the first class module values of OCaml, it is
easy for a PPX to have the same functionality as Modular Implicits and
type classes.

\end{abstract}

\section{Typeful PPX}\label{typeful-ppx}

\subsection{PPX + typing = Typeful PPX}\label{ppx-typing-typeful-ppx}

CamlP4 and PPX are preprocessing frameworks of OCaml and they are now
widely used to provide new functionalities to the language. For example,
syntax extension (ex. \texttt{pa\_monad} and \texttt{ppx\_moandic}) and
automatic code generation (ex. \texttt{deriving} and
\texttt{type\_conv}).
They are extremely useful in the real world programming, but untyped:
both CamlP4 and PPX are preprocessors which work over
\texttt{Parsetree}, not-yet-typed ASTs. Even if someone gets a tiny but
nifty idea of type dependent program transformation, it was almost out
of the scope of CamlP4 and PPX and it had to be implemented as a
compiler modification. Compiler modification is a very subtle to be done
correctly, especially if it involves with typing. Distributing and
installing compiler modifications are also hard: even with the help of
OPAM's \texttt{opam switch}, many OCaml users do not consider to invest
their time to try your modifications.

\emph{Typeful PPX} is a technique to overcome this difficulty of OCaml
language enhancement with types. It does not preprocess the input,
untyped AST of \texttt{Parsetree} directly, but type-check it firstly
and works on the typed AST of \texttt{Typedtree} to make use of type
annotations. Once the preprocessing of the typed AST is done, it untypes
the result to an untyped AST as the final output. From the point of view
of the host compiler which invokes a typeful PPX, it is just another
ordinary but rather complicated PPX which transforms untyped ASTs.
%
% \subsection{Pros and Cons}\label{pros-and-cons}
%
% Typeful PPX has the following benefits compared with the direct compiler
% modification:

Typeful PPX is type-safe. Its output is again type checked by the compiler.
Critical bugs in Typeful PPX should be found by this second type check.
% In the direct compiler modification, bugs in the typing layer tend to
% make the type system unsafe and they are hard to detect and fix. This
% should also make users feel much easier to try new functionalities via
% Typeful PPX than the compiler patching. If still unsure, users can
% always print out the final output and verify what Typeful PPX actually
% does.
It is easy to distribute, install and use. It is just a PPX.
Typeful PPXs are easily installable via the packaging system and users
can use their functions with their vanilla OCaml compiler immediately.
% 
% \item Easy future integration. Implemented as a transformer of
% \texttt{Typedtree}, future integration of Typeful PPXs into the real
% compiler modification can reuse the much of their code.
% \end{itemize}

% \noindent
% Unfortunately it has some drawbacks too:
% 
% \begin{itemize}
% \item Up-to typing layer: this is a typeful program transformation and
% therefore cannot change the lower details like code generation.
% 
% \item Not working with toplevel (REPL): PPX preprocessing works against each
% compilation unit, which is one toplevel expression in OCaml toplevel.
% Typeful PPX usually must keep various type informations across toplevel
% expressions therefore does not work in OCaml toplevel.
% \end{itemize}

\subsection{Technical details}\label{technical-details}

Typeful PPX requires to integrate the OCaml type checker to PPX
but the necessary tools are already available as OCaml's compiler API
library \texttt{compiler-libs}. Just slight modification of
\texttt{driver/compiler.ml} is required to wire up the inputs and
outputs of PPX, type checker and the typeful program transformation.
For typeful program transformation, \texttt{compiler-libs} already
includes \texttt{TypedtreeMap}, a module for an easy interface to build
a mapper. 
%Program transformation does not necessarily produce well-typed
%AST as its result, since it is get untyped. 
Untyping is also ready at
\texttt{tools/untypeast.ml} of OCaml compiler source.

\section{Ppx\_implicits}\label{ppxux5fimplicits}

\texttt{Ppx\_implicits}\footnote{ \texttt{https://bitbucket.org/camlspotter/ppx\_implicits} } is an example of Typeful
PPX, which provides type dependent \emph{implicit values}. It is also
extended to have implicit parameters\cite{scalaimplicits}, Modular Implicits\cite{ocamlimplicits}
and type class\cite{typeclass} like features as a PPX.

% \subsection{Simple implicit values\texttt{{[}\%imp M{]}}}\label{simple-implicit-values-imp-m}

The key feature of \texttt{ppx\_implicits} is type dependent
implicit values \texttt{[\%imp M]}, which are auto-generated
to match with the typing context by PPX, combining 
``instances'' defined in module \texttt{M}:
% For example, an
% annotated expression \texttt{{\it e}{[}@imp M{]}} is replaced with an
% expression of the same type of \textit{e}, using the instances available
% under module \texttt{M}. For example, here is a simple overloading of
% plus operators:

\begin{Shaded}
\begin{Highlighting}[]
  \OtherTok{module} \NormalTok{Show = }\KeywordTok{struct}
    \KeywordTok{let} \NormalTok{int}   \NormalTok{= Pervasives}\KeywordTok{.}\NormalTok{string_of_int}
    \KeywordTok{let} \NormalTok{float = Pervasives}\KeywordTok{.}\NormalTok{string_of_float}
    \KeywordTok{let} \NormalTok{list} \NormalTok{~_x:show xs = }\StringTok{"[ "} \NormalTok{^ String}\KeywordTok{.}\NormalTok{concat }\StringTok{"; "} \NormalTok{(List}\KeywordTok{.}\NormalTok{map show xs) ^ }\StringTok{" ]"}
    \CommentTok{(* Label starts with '_' has a special meaning in instances:}
  \CommentTok{     The arguments are generated recursively from the instance sets. *)}
  \KeywordTok{end}
  
  \KeywordTok{let} \NormalTok{() = }\KeywordTok{assert} \NormalTok{( }\NormalTok{[\%imp }\DataTypeTok{Plus}] \NormalTok{1 = }\StringTok{"1"} \NormalTok{)}       \CommentTok{(* replaced with Show.int *)}
  \KeywordTok{let} \NormalTok{() = }\KeywordTok{assert} \NormalTok{( }\NormalTok{[\%imp }\DataTypeTok{Plus}] \NormalTok{1.2 = }\StringTok{"1.2"} \NormalTok{)}   \CommentTok{(* replaced with Show.float *)}
  \KeywordTok{let} \NormalTok{() = }\KeywordTok{assert} \NormalTok{( [%imp }\DataTypeTok{Show}] \NormalTok{[1;2] = }\StringTok{"[ 1; 2 ]"} \NormalTok{)} \CommentTok{(* replaced with Show.(list ~_x:int )*)}
\end{Highlighting}
\end{Shaded}

\texttt{Ppx\_implicits} overload resolution is pretty normal: 
it uses type unification with backtracking.
% It tries to unify the type of implicit values and the type
% of each instance. If the unification fails, the instance is discarded.
% If there is only one match, the matched instance is used for the
% replacement. Otherwise, if none, \texttt{ppx\_implicits} fails due to no
% possible instance. If there are multiple matches, it also fails due to
% the ambiguity.
% \texttt{(assert false){[}@imp M{]}} is often used to generate a value
% which matches with its typing context. Therefore,
% \texttt{ppx\_implicits} has a sugar \texttt{{[}\%imp M{]}} for
% \texttt{(assert false){[}@imp M{]}}.
% Instances can be combined recursively.
%
%\begin{Shaded}
%\begin{Highlighting}[]
%  \OtherTok{module} \NormalTok{Show2 = }\KeywordTok{struct}
%    \OtherTok{include} \NormalTok{Show}
%    \KeywordTok{let} \NormalTok{list} \NormalTok{~_x:show xs = }\StringTok{"[ "} \NormalTok{^ String}\KeywordTok{.}\NormalTok{concat }\StringTok{"; "} \NormalTok{(List}\KeywordTok{.}\NormalTok{map show xs) ^ }\StringTok{" ]"}
%    \CommentTok{(* Label starts with '_' has a special meaning in instances:}
%  \CommentTok{     The arguments are generated recursively from the instance sets. *)}
%  \KeywordTok{end}
%  
%  \KeywordTok{let} \NormalTok{() = }\KeywordTok{assert} \NormalTok{( [%imp }\DataTypeTok{Show2}] \NormalTok{[1;2] = }\StringTok{"[ 1; 2 ]"} \NormalTok{)} \CommentTok{(* replaced with Show2.list ~_x:Show2.int *)}
%\end{Highlighting}
%\end{Shaded}
%
Instance constraint is expressed having a special label names 
start with \texttt{\_} to function arguments.

We can define \texttt{show} function at this level but it still requires
explicit dispatching of an implicit value:

\begin{Shaded}
\begin{Highlighting}[]
  \KeywordTok{let} \NormalTok{show (imp : 'a -> }\DataTypeTok{string}\NormalTok{) x = imp x}
  \KeywordTok{let} \NormalTok{() = }\KeywordTok{assert} \NormalTok{( show [%imp }\DataTypeTok{Show}] \NormalTok{[1; 2] = }\StringTok{"[ 1; 2 ]"} \NormalTok{)}
\end{Highlighting}
\end{Shaded}

\subsection{Implicit arguments as optional
arguments}\label{implicit-argument-using-optional-argument}

The dispatching arguments of implicit values should be omittable
in order to write the above code simply as \texttt{show {[}1; 2{]}}.
We can make use of OCaml's optional arguments for this purpose:

\begin{Shaded}
\begin{Highlighting}[]
  \KeywordTok{let} \NormalTok{unpack = }\KeywordTok{function} \DataTypeTok{None} \NormalTok{-> }\KeywordTok{assert} \KeywordTok{false} \NormalTok{| }\DataTypeTok{Some} \NormalTok{x -> x}
  \KeywordTok{let} \NormalTok{show ?imp x = unpack imp x}
  \KeywordTok{let} \NormalTok{() = }\KeywordTok{assert} \NormalTok{( show [1; 2] = }\StringTok{"[ 1; 2 ]"} \NormalTok{)}
\end{Highlighting}
\end{Shaded}

The expression \texttt{show {[}1; 2{]}} is equivalent with
\texttt{show ?imp:None {[}1; 2{]}}, and it must be transformed to
\texttt{show \textasciitilde{}imp:{[}\%imp Show{]} {[}1; 2{]}} 
%by \texttt{ppx\_implicits}. 
But without omitting
\texttt{{[}\%imp Show{]}}, how can we tell an optional argument is for
the implicit value dispatch for some instances \texttt{Show}? It
requires one more trick: we transfer the information to the type of the
optional argument:

\begin{Shaded}
\begin{Highlighting}[]
  \OtherTok{module} \NormalTok{Show2 = }\KeywordTok{struct}
    \KeywordTok{type} \NormalTok{'a __imp__ = }\DataTypeTok{Packed} \KeywordTok{of} \NormalTok{('a -> }\DataTypeTok{string}\NormalTok{)}
    \KeywordTok{let} \NormalTok{pack ~_x = }\DataTypeTok{Some} \NormalTok{(}\DataTypeTok{Packed} \NormalTok{_x)}
    \KeywordTok{let} \NormalTok{unpack = }\KeywordTok{function} \DataTypeTok{None} \NormalTok{-> }\KeywordTok{assert} \KeywordTok{false} \NormalTok{| }\DataTypeTok{Some} \NormalTok{(}\DataTypeTok{Packed} \NormalTok{x) -> x}
    \OtherTok{include} \NormalTok{Show}
  \KeywordTok{end}
  
  \CommentTok{(* val show : ?imp:'a Show2.__imp__ -> 'a -> string *)}
  \KeywordTok{let} \NormalTok{show ?imp x = Show2}\KeywordTok{.}\NormalTok{unpack imp x}
\end{Highlighting}
\end{Shaded}

Now \texttt{show} takes a value of \texttt{'a Show2.\_\_imp\_\_} as an
optional argument. Adding \texttt{ppx\_implicits} another rule to
transform function arguments of the form \texttt{?l:None} where
\texttt{None} has type \texttt{\_ PATH.\_\_imp\_\_ option} to
\texttt{?l:{[}\%imp PATH{]}}, \texttt{show {[}1; 2{]}} is properly
transformed to the following expression:

\begin{Shaded}
\begin{Highlighting}[]
  \NormalTok{show ?imp:Show2.(pack ~_x:(}\NormalTok{list} \NormalTok{~_x:int)) [1; 2]}
\end{Highlighting}
\end{Shaded}

\subsection{Instance search space configuration}\label{instance-search-space-configuration}

(This part is just roughly implemented in the current implementation.)

We have seen the simplest instance search policy of
\texttt{ppx\_implicits}: \texttt{{[}\%imp PATH{]}}, the values defined
inside a specific module path \texttt{PATH}. 
\texttt{ppx\_implicits} can support other instance search policies,
since the overload resolution itself is independent from 
the instance search space definitions.

For example, it is not difficult to use \texttt{open} directive to
accumulate instance space like Haskell's \texttt{import} and Modular
Implicits' \texttt{open implicit} declarations. For example,
\texttt{{[}\%imp "\textless{}opened\textgreater{}.Show"{]}} is to
instruct the PPX to gather instances from the sum-modules named
\texttt{Show} under modules opened by the \texttt{open} directive.

Another policy, which is not orthodox but probably very useful in the
current OCaml library structure, is a control based on the module names
occurring in implicit value types. For example, if an implicit
expression
\texttt{{[}\%imp "\textless{}related\textgreater{}.sexp\_of*"{]}} has a
type \texttt{PATH.t -\textgreater{} Sexplib.Sexp.t}, then instances are
the values whose paths match with \texttt{PATH.sexp\_of*} and
\texttt{Sexplib.Sexp.sexp\_of*}. Many existing library functions 
for data types which take sub-functions for their parameter types 
can be easily integrated as overloaded functions by this policy.

\subsection{To Modular Implicits and type
classes}\label{to-modular-implicits-and-type-classes}

Once the implicit arguments by optional arguments works, it is almost
trivial to have the same functionalities of Modular Implicits and type
classes in \texttt{ppx\_implicits}: it is to generate modules as first
class values at the caller site, then to make it back to normal modules
at the callee site.
Details are omitted for this proposal but you can see some examples at
\texttt{tests/typeclass.ml} of the source code.

\section{Conclusion}\label{conclusion}

Typeful PPX is a way to write type dependent extensions of OCaml. This
can be used not only for a fast and safe prototyping of OCaml compiler
modification but also for real use.

 
% -------
% 
% 1. Mikael Rittri. Using types as search keys in function libraries. Journal of Functional Programming, 1(1):71--89, 1991.

\begin{thebibliography}{9}

% \bibitem{lamport94}
%   Leslie Lamport,
%   \emph{\LaTeX: a document preparation system}.
%   Addison Wesley, Massachusetts,
%   2nd edition,
%   1994.

\bibitem{scalaimplicits}
  Oliveira, Bruno CdS, Adriaan Moors, and Martin Odersky. "Type classes as objects and implicits." ACM Sigplan Notices. Vol. 45. No. 10. ACM, 2010.

\bibitem{ocamlimplicits}
  Leo White, Fred\'eric Bour. "Modular implicits." ML Workshop, 2014.

\bibitem{typeclass}
  Wadler, Philip, and Stephen Blott. "How to make ad-hoc polymorphism less ad hoc." Proceedings of the 16th ACM SIGPLAN-SIGACT symposium on Principles of programming languages. ACM, 1989.

\end{thebibliography}


\end{document}
